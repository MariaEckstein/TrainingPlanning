\documentclass[11pt]{article} % For LaTeX2e
\usepackage{rldmsubmit,palatino}
\usepackage{graphicx}
\usepackage{amsmath}

\usepackage[style=apa,backend=biber]{biblatex}
\addbibresource{TwoStep.bib}
\DeclareLanguageMapping{english}{american-apa}

\title{Manipulating Model-based and Model-free Reinforcement Learning in Humans}


\author{
Maria K Eckstein \\
Department of Psychology \\
University of California, Berkeley \\
Berkeley, CA 94720 \\
\texttt{maria.eckstein@berkeley.edu} \\
\And
Klaus Wunderlich \\
Department of Psychology \\
Ludwig Maximilian University, Munich \\
Geschwister-Scholl-Platz 1, 80539 Munich \\
\texttt{klaus.wunderlich@lmu.de} \\
\And
Anne GE Collins \\
Department of Psychology\\
University of California, Berkeley \\
Berkeley, CA 94720  \\
\texttt{annecollins@berkeley.edu} \\
}

\newcommand{\fix}{\marginpar{FIX}}
\newcommand{\new}{\marginpar{NEW}}

\begin{document}

\maketitle

\begin{abstract}

When deciding what to do, humans and animals employ (at least) two different decision systems: oftentimes, we rely on habits, fixed stimulus-response associations, which have been shaped by past rewards, are computationally cheap, and enable fast responses ("model-free" decision making). But we can also---effortfully---make decisions using planning, mental simulation of different courses of actions and their outcomes, and selection of the one course that leads to the desired goal ("model-based" decision making). 

Previous research in humans has shown that it is possible to experimentally reduce model-based decision making relative to model-free decision making, for example by inducing stress (Otto et al., 2013). In the current study, we investigated whether it is also possible to increase model-based decision making. To do this, we implemented a cognitive intervention, which engaged participants in forward-planning and mental simulation (model-based condition), habitual, reward-based processes (model-free condition), or unrelated processes (active control). We assessed decision strategies using the 2-step task (Daw et al., 2011), and fitted a hybrid model-free/model-based reinforcement learning model to estimate participants' relative weight on each process. In accordance with our pre-registered predictions, we found that the model-based intervention increased the relative weight of model-based versus model-free decision strategies, whereas the model-free intervention had no effect. 

These results could have important practical benefits not only for vulnerable populations with known difficulties in decision making, but also for healthy persons who fall back to model-free habit under stress or time pressure, with negative consequences. Nevertheless, more research is needed to investigate the unexpected effect of the control intervention on decision making.

\end{abstract}

\keywords{
Reinforcement Learning Model-based Model-free Decision Making
}

\acknowledgements{This research was supported by Berkeley's XLab, which provided the testing space and awarded a grant covering participant payment. The research was also supported a graduate-student fellowship from the German Academic Exchange Service (DAAD).}


\startmain

\section{Introduction}

Humans make many decisions habitually: for example, we effortlessly navigate the route we take every day, following a fixed sequence of actions, and using landmarks to trigger subsequent actions. But humans also make decisions in a goal-directed way. For example, when we plan how to reach a new destination, we flexibly combine individual pieces into a new route, using a cognitive map or model of our environment. These two different modes of decision making, habitual and goal-directed, have long been differentiated in psychology, and form the basis of two different schools of thought, namely Behaviorism (e.g., \cite{skinner_why_1977}) and cognitivism (e.g., \cite{tolman_cognitive_1948}). A parallel differentiation between decision making strategies exists in reinforcement learning, a branch of machine learning. Here, the distinction is between model-based (MB; similar to goal-directed) and model-free (MF; similar to habitual) agents. MB agents use a model of the environment to simulate possible actions and outcomes, and then determine which actions are expected to lead to the best outcomes. In contrast, MF agents determine the value of actions by accumulating the past reward history of these actions.

Ever more often, reinforcement learning algorithms are applied in psychological research. This has led to the discovery that activity in the brain's dopaminergic "reward system" (\cite{wise_brain_1989}) coincides with the occurrence of reward prediction errors as specified in MF reinforcement learning (\cite{schultz_neural_1997}). MB learning, on the other hand, has been shown to rely on a distinct brain network including frontal cortex and dorsal striatum (\cite{dolan_goals_2013}). Human learning and decision making relies on both MB and MF processes (\cite{daw_model-based_2011}), and a key question is how we arbitrate between the two. Previous studies have shown that cognitively demanding MB decision making is less prevalent when time or cognitive resources are sparse, for example during stress (\cite{schwabe_stress-induced_2011}) or multi-tasking (\cite{otto_curse_2013}). On the other hand, no situations have yet been identified that increase MB decision making. The only study that has shown an increase was a pharmacological manipulation of dopamine levels (\cite{wunderlich_dopamine_2012}). 

In the current study, we therefore sought to investigate whether a cognitive intervention could increase MB decision making. Cognitive strategies are influenced by prior cognitive activities (\cite{jaeggi_short-_2011}; \cite{muraven_self-regulation_2000}). Thus, we predicted that we could affect MB decision processes by training participants on tasks involving forward-planning and mental simulation, whereas training on tasks involving habitual, reward-driven behavior should affect the MF process. Here, we test this prediction in a behavioral study. The study was pre-registered on the Open Science Framework prior to data collection (osf.io/nw9vz).

\section{Methods and Results}
\subsection{Study Design and Description of the Tasks}

116 participants took part in the two-session experiment. In session 1, all participants first performed the 2-step decision making task (\cite{daw_model-based_2011}, see description below; Figure \ref{TwoStep}B), then received one of three training interventions designed at testing our hypothesis, then were tested on the 2-step task again (run 2). Participants came back for a third assessment of the 2-step task 2 days later (run 3; Figure \ref{TwoStep}A).% first high level description of the protocol.

Participants were randomly assigned to one of three interventions: model-based (MB), model-free (MF), or control. We chose tasks that were well established in the literature for engaging cognitive and neural processes corresponding to each mode of decision making (MB and MF), or for not engaging MB or MF processes (control). For training, MB participants engaged in two tasks that were closely related to MB decision making: a planning-intensive version of the Tower of London task (\cite{beauchamp_dynamic_2003}), and a rule-based category learning task (\cite{maddox_dissociating_2004}). Both tasks engage model-based planning or cognitive control, and rely on the brain network required for MB decision making (\cite{dolan_goals_2013}). MF participants engaged in tasks targeted at MF processes: a habitual reward-based task (\cite{tricomi_value_2015}) and an information-integration category learning task (\cite{maddox_dissociating_2004}). Both tasks engage long-term information integration and habitual behaviors, and rely on brain regions underlying MF decisions (\cite{dolan_goals_2013}). Finally, the control tasks (number comparison task: \cite{piazza_neurocognitive_2010}; orientation discrimination task: \cite{sasaki_advances_2010}) were unrelated to MB or MF decision making. 

Participants' decision strategies were assessed using the 2-step task (\cite{daw_model-based_2011}; see Figure \ref{TwoStep}B). On each trial, this task involves two sequential binary decisions, which are potentially followed by a reward. The first decision determines with high probability what the second set of choices will be; however, in rare transition cases, the opposite set of choices is offered instead. This task is designed so that MB and MF strategies are distinguishable. A MF agent learns the values of a state-one action $a_{1}$ by accumulating the rewards obtained in trials in which $a_{1}$ was selected; future choices of $a_{1}$ are therefore mainly driven by past reward, independently of whether the trial included a common or rare transition between $s_{1}$ and $s_{2}$. A MB agent, on the other hand, selects $a_{1}$ based on its knowledge of the task structure, taking into account the transition probabilities between state $s_{1}$ and $s_{2}$ when reasoning which action $a_{1}$ to take. Future actions $a_{1}$ therefore depend on both past rewards and transition probabilities. Specifically, MB agents tend to repeat $a_{1}$ upon reward when the transition was common, but select the alternative $a_{1}$ upon reward when the transition was rare. The opposite pattern emerges in unrewarded trials.
%
\begin{figure}
	\includegraphics[width=\linewidth]{RLDMfigures/TwoStepab2.png}
	\caption{Experimental procedure (A) and 2-step task (B). (A) as described in the main text. (B) 2-step task: each trial has two successive states, $s_{1}$ and $s_{2}$. In $s_{1}$, participants decide between two actions $a_{1}$ (blue vs pink). One choice (e.g. blue) is followed by state $s_{2a}$ most of the time ("common transition"), the other one (pink) by state $s_{2b}$. After deciding between two actions in state $s_{2}$, participants receive a reward with probability $P(R|a_{2})$, which only depends on $a_{2}$. Reward probabilities change over time to ensure continued learning.}
	\label{TwoStep}
\end{figure}
%
\subsection{Behavioral Switch-Stay Analysis}
We tested 116 participants in this paradigm. Standard procedures were used for data cleaning and exclusion of participants, resulting in 303 datasets from 114 participants (110 in run 1: 44 MB, 43 MF, 21 control; 103 in run 2: 43 MB, 41 MF, 19 control; and 90 in run 3: 36 MB, 35 MF, 19 control). We first analyzed the 2-step task using logistic regression to assess the effect of rewards and transitions on future choices (\cite{akam_simple_2015}; see Figure \ref{Results}A and B). Participants repeated rewarded actions more often than unrewarded actions, a sign of MF behavior. This effect was statistically significant in all runs in the MF and MB groups, but not in the control group, as revealed by main effects of previous reward on staying, in logistic mixed-effects regression models controlling for choice repetition and key repetition (control group: all $\beta's < 0.12$, $z's < 1.61$, $p's > 0.11$; MB: all $\beta's > 0.15$, $z's > 3.79$, $p's < .001$; MF: all $\beta's > 0.18$, $z's > 3.41$, $p's < .001$). Besides this MF component, participants also showed markers of MB decision making, revealed by significant positive interactions between reward and transition, in run 2 (control: $\beta = 0.11$, $z = 2.05$, $p = .040$; MB: $\beta = 0.11$, $z = 2.54$, $p = .011$; MF: $\beta = 0.10$, $z = 2.92$, $p = .0035$). We then tested for differences between groups, using interaction contrasts. We found no differences in the MF or MB component for any run, shown by non-significant interactions between reward and group, all $\chi^{2}(2) < 1.33$, $p's > .40$, and between reward, transition, and group, all $\chi^{2}(2) < 1.17$, $p's > .56$. Lastly, we found that the model-based component changed over time, as revealed by the interaction between reward, transition, and run, $\chi^{2}(4) < 15.07$, $p = .0046$. However, this change over time did not differ by group, as revealed by the non-significant interaction of this effect with training, $\chi^{2}(16) < 6.04$, $p = .99$). Thus, it probably reflected practice, rather than intervention effects.  % Not sure if this last results is important - what do you think?

In summary, the MB and MF groups showed MF characteristics in all runs and additional MB components in run 2. The control group showed no sign of MF decision making, but MB decisions in run 2. These results were confirmed by regression models integrating a larger number of previous trials to predict actions (\cite{akam_simple_2015}; results not shown).

\subsection{RL Modeling Analysis}
The previous analyses focus on the influence of a single previous trial on decision making. We followed these analyses up with computational modeling to assess participants' decision strategies as a mixture of long-term MB vs. MF reinforcement learning (\cite{akam_simple_2015}; \cite{daw_model-based_2011}; \cite{sutton_reinforcement_2017}). We specified a hybrid model, in which agents determine action values by combining MB and MF value estimates. We then fit this model to each participant's actions by selecting parameter values that maximized the likelihood of the observed actions under the model. The parameter $w$, which determines the weight of MB and MF value estimates, was used to assess participants' decision strategies.

Our model was similar to previously published models of this task (e.g., \cite{wunderlich_dopamine_2012}). Specifically, agents update action values $Q$ for actions $a_{2}$ in the second state $s_{2}$ by observing the trial outcome (reward $R = 1$ or $R = 0$):
%
\begin{equation}
Q(s_{2}, a_{2}) = Q(s_{2}, a_{2}) + \alpha_{2} \cdot RPE,
\end{equation}
%
where the reward prediction error $RPE = R - Q(s_{2}, a_{2})$ and $\alpha_{2}$ is the agent's learning rate in the second state. The update of first-state action values $Q(s_{1}, a_{1})$ differs between MB and MF agents. MF agents use the outcome of $a_{1}$ to update $Q(s_{1}, a_{1})$. The outcome of $a_{1}$ consists in the value of the action chosen in $s_{2}$ and the trial's reward $R$.
%
\begin{equation}
Q_{mf}(s_{1}, a_{1}) = Q_{mf}(s_{1}, a_{1}) + \alpha_{1} \cdot (VPE + \lambda \cdot RPE),
\end{equation}
%
where the value prediction error $VPE = Q_{mf}(s_{2}, a_{2}) - Q_{mf}(s_{1}, a_{1})$. The weight of the RPE is determined by $\lambda$, a temporal discounting factor.

MB agents determine $Q(s_{1}, a_{1})$ based on an internal predictive model, taking into account the transition probability $p(s_{2}, a_{1}, s_{1})$ between states $s_{1}$ and $s_{2}$ upon choosing $a_{1}$, and planning to select the best available action $a_{2}$ thereafter:
%
\begin{equation}
Q_{mb}(s_{1}, a_{1}) = \sum_{s_{2}} p(s_{2}, a_{1}, s_{1}) \cdot max(Q(s_{2}, a_{2})).
\end{equation}
%
Agents combine MB and MF value estimates using a weighted average, $Q_{hyb}(s_{1}, a_{1}) = (1 - w) \cdot Q_{mf}(s_{1}, a_{1}) + w \cdot Q_{mb}(s_{1}, a_{1})$. The parameter $w$ determines the weight of MB versus MF values. Agents select actions $a$ according to a softmax decision rule, which takes into account the action's value $Q(s, a)$, but also whether the same action was taken in the previous trial (choice perseverance $p$) and whether the same key was used to select it (key perseverance $k$). The inclusion of $k$ is an extension of previous models and improved the model fit significantly. We validated our model by simulating agents with different parameter values and subsequently recovering these parameters (data not shown).

We then aimed to characterize human performance in terms of the model parameters, specifically parameter $w$ indicating the balance between MB and MF decision making. Model comparison with Bayes Information Criterion (BIC) indicated that a full hybrid model albeit with fixed future discounting $\lambda=1$ was best (Wilcoxon signed-ranks test, $w = 61964$, $p < .001$). In accordance with the previous analyses, model fitting results (Figure \ref{Results}B) showed that $w$ increased from run 1 (before intervention) to run 2 (right after intervention) in the MB group, as revealed by a repeated-measures t-test, $t(36) = 2.25$, $p = 0.031$. This was not the case in the MF or control group (MF: $t(36) = 0.80$, $p = 0.43$; control: $t(15) = 0.46$, $p = 0.65$). Nevertheless, no long-term effects of the training were evident (all $ts < 0.24$, all $ps > .81$). A mixed-effects regression model supported these results. We tested for effects of group and run on $w$ (group, contrast of MB versus MF and control: $\beta = 0.036$, $p = .039$; interaction contrast with run: $\beta = 0.063$, $p = .033$).
%
\begin{figure}
	\includegraphics[width=\linewidth]{RLDMfigures/Results.png}
	\caption{Results of the 2-step analyses. (A) Participants' probability of repeating the action $a_{1}$ taken in the previous trial ("s1 stay probability"), as a function of reward and transition in the previous trial, including standard errors (SE). (B) Beta weights of logistic regression predicting staying (repeating previous $a_{1}$) from previous reward, transition, and their interaction. Empty circles: $p >= 0.05$; filled circles: $p < 0.05$. (C) Means and SE of fitted parameter $w$.}
	\label{Results}
\end{figure}
%
\section{Conclusion}
In this study, we aimed at investigating which factors might influence human decision making. Specifically, we tested whether the use of MB versus MF strategies depends on previous cognitive activities, such that forward planning and rule-based category learning would increase MB decision strategies, whereas habitual, reward-driven stimulus-response behavior would increase MF decision making. In accordance with our hypotheses, we found that decision strategies became more model-based after the MB intervention involving forward planning and rule-based category learning, whereas the MF and control interventions had no effect.

%This result is particularly interesting because the training tasks were conceptually very different from the decision-making task, which we used to assess the degree of model-basedness. "Far" transfer like this is less commonly observed than near tranfer between very similar tasks (\cite{jaeggi_short-_2011}).

These results support our hypothesis that model-based decision making is closely related to forward-planning and rule-based category learning. Indeed, forward-planning and model-based decision making both rely on the employment of cognitive models. We have shown that creating and employing cognitive models in one task facilitates subsequent model-based reasoning in a completely unrelated task. In other words, humans seem to be able to learn model use on a highly abstract level. Rather than relying more on one specific model or one kind of models, participants learned to rely more on cognitive models in general.

Transitioning to rule-based category learning, previous research has already suggested similarities to model-based reasoning, both with respect to underlying cognitive processes and brain areas (Maddox \& Ashby, 2004). Nevertheless, this assumption had not yet been tested experimentally. Our findings support the claim that both kinds of cognitive activities indeed engage similar processes, despite their superficial differences. Future research will be needed to assess the exact overlap. 

Another open question for future research is how the effects of forward-planning and rule-based category learning combined in producing the observed increase in model-based decision making. It is possible that the effects were driven by just one of the training tasks or that both interacted to produce the results. Having established that it is possible to increase model-based decision making with a combination of both tasks, future studies will be needed to disentangle the contributions of either one.

%Surprisingly, we found that participants in the MB and MF groups developed an unexpected decision pattern during training: in run 3, participants were more likely to repeat actions that were followed by common rather than rare transitions, as revealed by a significant interaction contrast between transition and group (control vs active), $\beta = 0.041$, $z = 2.23$, $p = .026$. This effect is not expected from either MB or MF decision making. Instead, it might reflect a preference for choices that lead to predicted outcomes. However it is unclear why this preference arose in the active groups, but not in the control group. Such a pattern has not been observed in the previous literature and might be interesting to pursue in future research.
% Did you check in other 2-step papers if they have main effects of transition as we observe here? -> I haven't seen it in any other paper I checked.

Summing up, research into the manipulation of MB and MF decision strategies is of great relevance. MF decisions are fast and efficient, but can be suboptimal in the long term (eating the chocolate cake). MB decisions (preparing for an exam) require more effort but can often lead to better long-term outcomes. Many people, including persons with clinical conditions such as ADHD, depression, or eating disorders, would benefit from a cognitive intervention that facilitates MB decision making and trains selecting appropriate decision strategies. Further research is needed to establish the usefulness of interventions similar to the one presented here in clinical populations.

\section{References}

\printbibliography

\end{document}
