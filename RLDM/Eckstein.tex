\documentclass[11pt]{article} % For LaTeX2e
\usepackage{rldmsubmit,palatino}
\usepackage{graphicx}
\usepackage{amsmath}

\title{Manipulating Model-based and Model-free Reinforcement Learning in Humans}


\author{
Maria K. Eckstein \\
Department of Psychology \\
University of California, Berkeley \\
Berkeley, CA 94720 \\
\texttt{maria.eckstein@berkeley.edu} \\
\And
Klaus Wunderlich \\
Department of Psychology \\
Ludwig Maximilian University, Munich \\
Geschwister-Scholl-Platz 1, 80539 Munich \\
\texttt{klaus.wunderlich@lmu.de} \\
\And
Anne Collins \\
Department of Psychology\\
University of California, Berkeley \\
Berkeley, CA 94720  \\
\texttt{annecollins@berkeley.edu} \\
}

\newcommand{\fix}{\marginpar{FIX}}
\newcommand{\new}{\marginpar{NEW}}

\begin{document}

\maketitle

\begin{abstract}
Abstract. 
\end{abstract}

\keywords{
Reinforcement Learning Model-based Model-free Decision Making
}

\acknowledgements{This research was supported by Berkeley's XLab, which provided the testing space and awarded a grant covering participant payment as well as by a fellowship from the German Academic Exchange Service (DAAD).}  


\section*{TDs:}
\begin{itemize}
	\item Do some more research on mb / mf studies
	\item Abstract
	\item Rescale p and k
	\item Simulate data using participants' parameter values -> recover
\end{itemize}

\startmain

\section{Introduction}

Animals as well as humans make many habitual decisions: for example, humans effortlessly follow the same path that they take every day, they follow fixed sequences of actions and use landmarks on the way as triggers for subsequent actions. But animals and humans also make goal-directed decisions. For example when planning to reach a goal, they can flexibly combine pieces of knowledge into a new route, based on a cognitive map, or model, of their environment. These two different modes of decision making, habitual and goal-directed, have long been differentiated in psychology, and form the basis of two very different schools of thoughts, namely Behaviorism (Watson; Skinner; Pavlov) and cognitivism (Tolman; Chomsky; Anderson; Miller). A parallel differentiation exists in computational models of reinforcement learning in machine learning. Here, the distinction is between model-based (MB; similar to goal-directed) and model-free (MF; similar to habitual) agents.  MB agents use a model of the environment to simulate possible actions and outcomes, and determine which action has the largest expected reward based on previous experience. In contrast, MF agents simply accumulate reward history into cached values of choices. 

Recently, reinforcement learning algorithms have been used in psychology to characterize the decision process more precisely and to identify the brain regions involved. A major finding concerns the striking similarity between activity in the dopamine system, the brain's "reward system" (Wise, 1982), and the occurrence of reward prediction errors as specified in MF reinforcement learning models (Schultz, 1997). In contrast, MB learning relies on a distinct brain network (including frontal cortex and dorsal striatum; Dolan \& Dayan, 2013). Human learning and decision making relies on both MB and MF in parallel (Daw 2011), and a key question is how we arbitrate between the two processes. Human studies have shown that cognitively demanding model-based decision making, is less prevalent when time and/or cognitive resources are sparse, for example during stress (Schwabe \& Wolf, 2011) or when multi-tasking (Otto, Gershman, Markman, \& Daw, 2013). On the other hand, no situations have yet been identified that increase model-based decision making, although t. The only study that has shown an increase in model-based decision making was a pharmacological manipulation of dopamine levels (Wunderlich, Smittenaar, \& Dolan, 2012). 

In the current study, we therefore sought to investigate whether a cognitive intervention could increase MB decision making. We hypothesized that decision strategies are influenced by prior cognitive activities. Forward-planning and mental simulation should influence the MB process, whereas habitual, reward-driven behavior should influence the MF process. This influence could go either way, fostering (cognitive enhancement hypothesis; CITE Jaeggi?) or depressing (cognitive depletion hypothesis; CITE) the subsequent decision strategy. An relative increase of MB vs. MF decision making after MB activities would support the cognitive enhancement hypothesis; the same increase after model-free activities would support the cognitive depletion hypothesis. We pre-registered these hypotheses on the Open Science Framework prior to data collection (www.osf.io/).

\section{Methods and Results}
\subsection{Study Design and Description of the Tasks}

116 participants took part in a two-session experiment (Figure \ref{TwoStep}A). All participants performed the 2-step decision making task (Daw 2011, see description below; Figure \ref{TwoStep}B), then received one of three training interventions, designed at testing our hypothesis, then were tested on the 2-step task again; they came back for a third assessment of the 2-step task 2 days later.% first high level description of the protocol.

Participants were randomly assigned to one of three groups with different training interventions: model-based intervention (MB), model-free intervention (MF), and control. We chose tasks that were well established in the literature for engaging the kind of cognitive and neural processes corresponding to each hypothesis (MB and MF), or for not engaging either (Control). For training, MB Participants engaged in two tasks that were closely related to model-based decision making: a planning-intensive version of the Tower of London task (CITE), and a rule-based category learning task (Ashby \& Maddox). Both tasks engage planning or cognitive control mechanisms, and rely on the control network required for MB decision making (cite).%The Tower of London task is model-based because it requires participants to use an internal model of the task to simulate potential moves and their outcomes, and to select the best sequence of moves. The rule-based category task has been shown to rely on model-based mechanisms, for example in that learning is unaffected by delayed feedback (Ashby and Maddox). Indeed, both Tower of London and the rule-based category learning task engage brain regions similar to model-based decision making (Beauchamp, Dagher, Aston \& Doyon, 2003; Ashby and Maddox).
% you can keep the details if there's room, but it's probably not needed if you're lacking space.
MF participants engaged in tasks targeted at model-free processes: a habitual reward-based task (Tricomi) and an information-integration category learning task (Ashby and Maddox). Both tasks engage long-term information integration habitual behaviors, and relay on brain regions underlying model-free decisions (cite)%The habitual reward-based task is a classical example of model-free, habitual learning, where participants were rewarded when they pressed the correct button in response to simple visual stimuli. The information-integration category task relies on model-free processes, as is evident in that performance is negatively affected by interventions that disrupt model-free learning, such as lack of immediate feedback. Again, both tasks have been shown to engage brain regions underlying model-free decisions. 
Finally, the control tasks were unrelated to MB or MF decision making. [AC do the same for this one].
%The number comparison task engages the approximate number sense and X and Y brain regions (Piazza, 2010). The orientation discrimination task involves perceptual learning and early visual processing and is associated with activity in the retinotopic region of V1 corresponding to the location of stimulus presentation (Kim, Ling, Watanabe, 2015). 

\begin{figure}
	\includegraphics[width=\linewidth]{TwoStepab2.png}
	\caption{Experimental procedure (A) and 2-step task (B). (A) In session 1, participants' decision strategies were assessed using the 2-step task, then of the three cognitive interventions followed, then participants did the 2-step task again. Two days later, participants came in for session 2 and did a third assessment of the 2-step task. (B) 2-step task: each trial has two successive states, $s_{1}$ and $s_{2}$. In $s_{1}$, participants decide between two actions $a_{1}$ (blue vs pink). One choice (e.g. blue) is followed by state $s_{2a}$ most of the time ("common transition"), the other one (pink) by state $s_{2b}$. In state $s_{2}$ participants decide between two actions and receive a reward with probability $P(R|a_{2})$ determined by $a_{2}$. Reward probabilities change over time to ensure continued learning.}
	\label{TwoStep}
\end{figure}

Participants' decision strategies were assessed using Daw et al.'s (cite) 2-step task (see Figure \ref{TwoStep}B). On each trial, this task involves two sequential binary decisions, followed by reward which drifts over time, to enforce learning. The first decision determines with high probability what the second set of two decision will be; however, in rare transition cases, the opposite set of decisions is offered instead. This task is designed so that MB and MF strategies are distinguishable. A MF agent learns the values of a state-one action $a_{1}$ by accumulating the rewards obtained in trials in which $a_{1}$ was selected; future first choices are therefore mainly driven by past reward, independently of whether the trial included a frequent or rare transition to the second decision stage. A MB agent, on the other hand, selects $a_{1}$ based on its knowledge of the task structure, taking into account the transition probabilities between first decision state $s_{1}$ and second decision state $s_{2}$ when reasoning which first action $a_{1}$ to take. Future actions therefore depend on both past rewards and the transition probability of rewarded trials. Specifically, [explain the expected interaction here to make it easier to describe results later: repeat trial rewarded in common transitions, but opposite in rare transitions].

\subsection{Behavioral Switch-Stay Analysis}
%The 2-step task can be analyzed in two ways. Behavioral analyses assess the influence of past rewards on participants' subsequent decisions (model-free characteristic) and they asses whether participants differentiate between common and rare transitions (model-based characteristic). RL modeling tests for model-based and model-free decisions explicitly by fitting RL models to the human data and estimating the weight of model-based and model-free decision making.

We tested XX participants, and used standard procedures to exclude participants (CITE). The resulting data consisted of 303 datasets from 114 participants (110 in run 1---44 MB, 43 MF, 21 control; 103 in run 2---43 MB, 41 MF, 19 control; and 90 in run 3---36 MB, 35 MF, 19 control). We first analyzed the 2-step task results with a logistic regression analysis assessing the effect of rewards and transitions on future choices ( cite; see Figure \ref{Results}A). Participants tended to repeat rewarded actions more often than unrewarded actions, a sign of model-free behavior. This effect was statistically significant in all runs in the MF and MB groups, but not in the control group, as revealed by main effects of reward in the previous trial on staying in logistic mixed-effects regression models including the factors choice repetition and key repetition, INSERT STATS. Besides this model-free component, participants also showed model-based decision making in that they differentiated between common and rare transitions. This was reflected in the interaction between reward and transition in runs 2 and 3 in the control group, run 2 in the MB group, and in runs 2 and 3 in the MF group (although the pattern was reversed here in run 3, so should not be taken as support for a model-based strategy), STATS. We then tested for differences between groups, using interaction contrasts. We found that the model-free component did not differ between groups in any run, revealed by non-significant interactions between reward and group, STATS. The model-based component, on the other hand, differed in run 3, as revealed by a significant interaction between reward, interaction, and group, STATS. [say what causes this difference]

In summary, the control group showed predominantly model-based behavior in runs 2 and 3. The MB and MF groups, on the other hand, showed model-free characteristics in all runs and additional model-based components in run 2 only. These analyses were also supported by regression models on the influence of outcomes further in the past and by logistic regression models on individual participants (not shown). The results were similar when only including data from participants who performed well on the training tasks.

\subsection{RL Modeling Analysis}
The previous analyses model choices based only on the characteristics of a single previous trial. Thus, we also used computational model to assess participants' decision strategies as a mixture of MB vs. MF reinforcement learning [cite]. We specified a hybrid model, in which agents determine action values by combining model-based and model-free value estimates. We then fit this model to each participant's actions by selecting parameter values that maximized the likelihood of the actions under the model. The parameter $w$, which determined the weight of model-based and model-free value estimates, was used to assess participants' decision strategy.

Our model was similar to the ones previously described for the 2-step task (CITE). In specific, agents updated action values $Q$ for actions $a_{2}$ available in the second stage by observing the trial outcome (reward $R = 1$ or $R = 0$), as follows:

\begin{equation}
Q(s_{2}, a_{2}) = Q(s_{2}, a_{2}) + \alpha_{2} \cdot RPE,
\end{equation}

whereby the reward prediction error $RPE = R - Q(s_{2}, a_{2})$ and $\alpha_{2}$ is the agent's learning rate in the second state.

The update of first-stage action values $Q(s_{1}, a_{1})$ differed between model-based and model-free agents. Model-free agents used the outcome of $a_{1}$ to update $Q(s_{1}, a_{1})$. The outcome of $a_{1}$ consists in the value of the action chosen in $s_{2}$ and the trial's ultimate outcome ($R = 1$ or $R = 0$).

\begin{equation}
Q_{mf}(s_{1}, a_{1}) = Q_{mf}(s_{1}, a_{1}) + \alpha_{1} \cdot (VPE + \lambda \cdot RPE),
\end{equation}

whereby the value prediction error $VPE = Q_{mf}(s_{2}, a_{2}) - Q_{mf}(s_{1}, a_{1})$. The weight of the RPE for first-stage updating was determined by $ \lambda $, a temporal discounting factor. The model with $\lambda = 1$ was the best-fitting model for our participants, suggesting that participants weighted RPEs equally strong as VPE when updating action values $Q_{mf}(s_{1}, a_{1})$.

Model-based agents determined $Q(s_{1}, a_{1})$ based on an internal predictive model, taking into account the transition probability $p(s_{2}, a_{1}, s_{1})$ between state $s_{1}$ and state $s_{2}$, when choosing action $a_{1}$:

\begin{equation}
Q_{mb}(s_{1}, a_{1}) = \sum_{s_{2}} p(s_{2}, a_{1}, s_{1}) \cdot max(Q(s_{2}, a_{2}))
\end{equation}

Agents combined model-based and model-free value estimates using a weighted average, $Q_{hyb}(s_{1}, a_{1}) = (1 - w) \cdot Q_{mf}(s_{1}, a_{1}) + w \cdot Q_{mb}(s_{1}, a_{1})$, whereby the parameter $w$ determined the weight of model-based versus model-free values. Agents selected actions $a$ according to a softmax decision rule, which took into account the actions' values $Q(s, a)$, but also whether the same action had been taken in the previous trial (choice perseverance $p$) and whether the same key had been pressed in the previous trial (key perseverance $k$). The inclusion of key perseverance is an extension of previous models and improved the model fit significantly. %Individuals' softmax temperature with respect to action values $Q(s, a)$ was determined by parameters $\beta_{1}$ and $\beta_{2}$ for the fist and second state. 
We validated our model by simulating agents with different parameter values and subsequently recovering these parameters (data not shown). %We also confirmed in simulations that reduced models, with certain parameters fixed, were best recovered by models with the same free parameters, rather than different models. [AC deleted for space]

\begin{figure}
	\includegraphics[width=\linewidth]{Results.png}
	\caption{Results of the 2-step analyses. (A) Participants' probability of repeating action last $a_{1}$ of the last trial ("s1 stay probability"), as a function of reward and transition in the previous trial. (B) Means and standard errors of model parameters fit to individual datasets. a1: $\alpha_{1}$, a1: $\alpha_{2}$, b1: $beta_{1}$, b2: $\beta_{2}$.}
	\label{Results}
\end{figure}

We then aimed to characterize human performance in terms of the model parameters, and specifically with parameter $w$ indicating the balance between model-based and model-free decision making. When fitting the model to human data, model comparison with Bayesian Information Criterion (BIC) indicated that a full hybrid model with fixed future discounting $\lambda=1$ was best (sign tests, STATS). %Matlab's fmincon function was used with 30 different start values to find the best-fitting parameters for each dataset. 
In accordance with the previous analyses, model fitting results (Figure \ref{Results}B) show that $w$ increased in run 2 in the MF group. Nevertheless, this effect was not statistically significant in a mixed-effects regression model testing for effects of group and run on $w$ (SHOULD I TEST GROUPS INDIVIDUALLY INSTEAD, COMPARING RUN 1 TO RUN 2 AND RUN 1 TO RUN 3?).[AC: no, it wouldn't be convincing at this point.]

\section{Conclusion}
In this study, we aimed at investigating factors that influence human decision making. In specific, we tested whether the use of model-based versus model-free strategies depended on previous cognitive activities, such that engagement in mental simulation and forward planning would influence model-based decision strategies, whereas habitual, reward-driven stimulus-response behavior would influence model-free decision making. Contrary to our hypothesis, we found that model-based decision making was highest in the control group, and potentially increased slightly in MB and MF in run 2, directly after the training. We found no long-term changes in decision strategy (run 3). (NOTE: SHOULD I TEST RUN 1 VS RUN 2 AND RUN 1 VS RUN 3 DIRECTLY? RATHER THAN PUTTING RUN AS A FACTOR INTO THE REGRESSION? [AC: no, this is more rigorous.]).

Looking at the behavioral data (Figure \ref{Results}A), the biggest differences seemed to arise between the control group and the two active groups. The decision pattern looks model-based as expected in the control group. However, MB and MF participants show an unexpected pattern in run 3: they are more likely to repeat an action that lead to a common rather than a rare transitions. This effect is not predicted by either model-based or model-free decision making. Instead, it might reflect a preference for choices that lead to predicted outcomes. However it is unclear why this should happen in active groups, but not control; this also hasn't been observed in previous literature. It might be interesting to pursue in future research. [AC Rephrase the last couple sentences, but this looks like a surprising effect, so you can try to spin that as a positive result in the study, and try to speculate why.]%But this does not match up with the modeling results...? [AC it can't be explained by those models]

One reason for our negative result might be that previous cognitive activities only carry over very briefly into subsequent activities, such that the strategy in the whole 2-step task, which took approximately 20 minutes, was not affected throughout. If this is the case, a shorter assessment of decision strategy might find something. In addition, there were differences between the design of the pilot study, which had initially suggested short-term cognitive depletion and long-term cognitive enhancement. Namely, participants only did the 2-step task twice in the pilot study, at the end of the training session and in the 1-week follow-up. Plus, the 2-step training was interleaved with the intervention tasks, potentially increasing carry-over effects. In the current study, the 2-step task was the first task of all, potentially leading participants to determine their personal strategy and being less flexible in changing it.

Research into the manipulation of model-based and model-free decision strategies is of great relevance. Model-free decisions are fast and efficient, but may be suboptimal in the long term (eating the chocolate cake). Model-based decisions (preparing for an exam) require more effort but often lead to better long-term outcomes when knowledge about environmental contingencies is relevant. Many vulnerable subjects, including persons with ADHD, depression, schizophrenia, Parkinson's disease, obesity, impulsitiy, eating disorders, or addiction would benefit from a cognitive intervention that facilitates model-based decision making and creates some ease in selecting and switching between both strategies, according to the context. Further research will be needed to establish how MB behavior can be encouraged in decision making.

\section{References}


\end{document}
